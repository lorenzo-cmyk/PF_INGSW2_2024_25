\chapter{Specific Requirements}
\label{chap:specific-requirements}

\par This chapter provides detailed specifications for the requirements previously mentioned that may require
additional clarification for the development team's implementation.

\section{External Interface Requirements}
\label{sec:external-interface-requirements}

\subsection{User Interfaces}
\label{subsec:user-interfaces}

\par The user interface of S\&C is implemented as a web-based application, making it accessible through standard web
browsers. No specialized software installation is required - users only need a web browser and an internet connection
to access all platform features.

\subsection{Hardware Interfaces}
\label{subsec:hardware-interfaces}

\par S\&C – being a web application – will be available on all devices, connected to the internet, capable of running a
modern browser. Users can use S\&C on laptops, desktops, smartphones, tablets, ...

\subsection{Software Interfaces}
\label{subsec:software-interfaces}

\par The system will primarily function as a standalone application. The only external dependency is the SSO portals of
the various UN organizations that have requested enrollment in S\&C, which will provide access to UN personnel and STs.

\subsection{Communication Interfaces}
\label{subsec:communication-interfaces}

\par The system primarily relies on HTTPS for secure communication. This includes communication between users and the
frontend, as well as between the frontend and backend, and between the backend and the various SSO portals.

\par The SMTP protocol is also used to send notifications to the user's email address when necessary.

\section{Functional Requirements}
\label{sec:functional-requirements}

\subsection{Requirements List}
\label{subsec:requirements-list}

% Define a new counter to add an unique ID to each requirement.
\newcounter{requirementCounter}
\setcounter{requirementCounter}{0}
% Define a new command to increment the counter and return the ID (R01, R02, ...).
\newcommand{\nextRequirementID}{
    \stepcounter{requirementCounter}R\ifnum\value{requirementCounter}<10 0\fi\arabic{requirementCounter}
}

\begin{longtable}{|l|p{0.9\textwidth}|}
    \hline
    \textbf{ID}        & \textbf{Requirement}                                                                                             \\
    \hline \hline
    \nextRequirementID & S\&C integrates with UN single sign-on (SSO) systems for authentication.                                         \\
    \hline
    \nextRequirementID & S\&C enables authorized internal staff to register users as UN representatives.                                  \\
    \hline
    \nextRequirementID & S\&C enables authorized internal staff to create login credentials for CO representatives.                       \\
    \hline
    \nextRequirementID & S\&C supports new ST registration through their UN's SSO system.                                                 \\
    \hline
    \nextRequirementID & S\&C enables existing ST users to log in through their UN's SSO system.                                          \\
    \hline
    \nextRequirementID & S\&C enables registered CO representatives to log in using their assigned credentials.                           \\
    \hline
    \nextRequirementID & S\&C enables registered UN representatives to log in through their UN's SSO system.                              \\
    \hline
    \nextRequirementID & S\&C enables STs to upload and store their CVs on the platform.                                                  \\
    \hline
    \nextRequirementID & S\&C provides automated CV feedback and improvement suggestions to STs.                                          \\
    \hline
    \nextRequirementID & S\&C allows STs to browse all publicly available internship opportunities.                                       \\
    \hline
    \nextRequirementID & S\&C provides filtering capabilities (by keyword, company, etc.) for internship searches.                        \\
    \hline
    \nextRequirementID & S\&C implements a recommendation system to suggest relevant internships to STs.                                  \\
    \hline
    \nextRequirementID & S\&C sends email notifications to STs about recommended internships.                                             \\
    \hline
    \nextRequirementID & S\&C displays comprehensive internship information for each listing.                                             \\
    \hline
    \nextRequirementID & S\&C integrates CO profiles within internship listing details.                                                   \\
    \hline
    \nextRequirementID & S\&C enables STs to submit internship applications before posted deadlines.                                      \\
    \hline
    \nextRequirementID & S\&C notifies STs when COs request questionnaire completion.                                                     \\
    \hline
    \nextRequirementID & S\&C allows STs to complete CO-specific questionnaires when requested.                                           \\
    \hline
    \nextRequirementID & S\&C provides a system for STs to submit concerns or complaints during their internship.                         \\
    \hline
    \nextRequirementID & S\&C notifies STs of any resolution or changes resulting from their submitted complaints.                        \\
    \hline
    \nextRequirementID & S\&C facilitates post-internship feedback collection from STs.                                                   \\
    \hline
    \nextRequirementID & S\&C enables COs to create and publish internship listings.                                                      \\
    \hline
    \nextRequirementID & S\&C allows COs to specify application deadlines for internship positions.                                       \\
    \hline
    \nextRequirementID & S\&C provides automated suggestions to improve internship listing content.                                       \\
    \hline
    \nextRequirementID & S\&C displays internship applications with anonymized CV access for COs.                                         \\
    \hline
    \nextRequirementID & S\&C enables COs to create evaluation questionnaires for applicants.                                             \\
    \hline
    \nextRequirementID & S\&C allows COs to distribute questionnaires to selected candidates.                                             \\
    \hline
    \nextRequirementID & S\&C provides COs access to completed questionnaire responses.                                                   \\
    \hline
    \nextRequirementID & S\&C enables COs to update the status of their internship listings.                                              \\
    \hline
    \nextRequirementID & S\&C provides a system for COs to submit concerns or complaints during internships.                              \\
    \hline
    \nextRequirementID & S\&C notifies COs of any resolution or changes resulting from their submitted complaints.                        \\
    \hline
    \nextRequirementID & S\&C facilitates post-internship feedback collection from COs.                                                   \\
    \hline
    \nextRequirementID & S\&C enables UNs to monitor the status of all its STs internships.                                               \\
    \hline
    \nextRequirementID & S\&C provides UNs access to feedback of their STs internship from both COs and STs, including submitted details. \\
    \hline
    \nextRequirementID & S\&C notifies UNs when new feedback is submitted by any party.                                                   \\
    \hline
    \nextRequirementID & S\&C allows UNs to block COs that violate platform guidelines or trust.                                          \\
    \hline
    \nextRequirementID & S\&C enables UNs to modify the status of internship positions.                                                   \\
    \hline
    \caption{Requirements Table}
    \label{tab:requirements-table}
\end{longtable}

\subsection{Use Case Diagrams}
\label{subsec:use-case-diagrams}

% Define a new counter to add an unique ID to each requirement.
\newcounter{useCaseCounter}
\setcounter{useCaseCounter}{0}
% Define a new command to increment the counter and return the ID (R01, R02, ...).
\newcommand{\nextUseCaseID}{
    \stepcounter{useCaseCounter}UC\ifnum\value{useCaseCounter}<10 0\fi\arabic{useCaseCounter}
}


% Don't now if it's better to how to format the "title" for the use case diagrams.
% Format to fix, but it's a start.

\nextUseCaseID: Student Create A Complaints

\par \textbf{Actors:} Student

\par \textbf{Entry Condition:} The ST is correctly logged in. The ST had decided to create a complaint about the internship he's doing.

\par \textbf{Flow of Events:}

\begin{enumerate}
    \item ST clicks on the "Create Complaint" button.
    \item S\&C send to the ST a form to fill with the complaint details.
    \item ST fills the form and submits it.
    \item S\&C stores the complaint in the database.
    \item S\&C request to the Emailing system to send an email to the UN about a new complaint about an internship of one of it's ST.
    \item The Emailing system sends the email to the UN.
    \item S\&C notifies the ST that the complaint was successfully submitted.

    Executed When UN respond:
    \item UN acknowledges the complaint.
    \item S\&C change the status of the complaint to "Acknowledged".
\end{enumerate}

\par \textbf{Exit Condition:} ST receive the confirmation of the complaint being saved correctly

\par \textbf{Exceptions:}  S\&C generated an internal error.

\begin{figure}[H]
    \centering
    \includegraphics[width=1.0\textwidth]{Images/UC_6.pdf}
    \caption{Student creates a complaint use case diagram}
    \label{fig:use-case-diagram-6}
\end{figure}


\nextUseCaseID: Company Login

\par \textbf{Actors:} Company
\par \textbf{Entry Condition:} CO has received the credentials by S\&C staff after the payment for the usage of the Website

\par \textbf{Flow of Events:}

\begin{enumerate}
    \item CO search the URL of S\&C login page.
    \item S\&C send to CO the login web page.
    \item CO inserts the username and password received.
    \item S\&C validates the credentials.
    \item S\&C sends the CO to the CO dashboard.
\end{enumerate}

\par \textbf{Exit Condition:} CO is redirected to the CO dashboard.

\par \textbf{Exceptions:}  
\begin{itemize}
    \item The credentials are wrong.
    \item The CO is was blocked by the S\&C staff after a violation of the platform guidelines.
    \item S\&C generated an internal error.
\end{itemize}

\begin{figure}[H]
    \centering
    \includegraphics[width=0.8\textwidth]{Images/UC_8.pdf}
    \caption{Company login use case diagram}
    \label{fig:use-case-diagram-8}
\end{figure}

\nextUseCaseID: Company Select Applicant Student For His Internship

\par \textbf{Actors:} Company

\par \textbf{Entry Condition:} CO is correctly logged in, has already created an internship listing, and has received applications from STs.
CO has already loaded the questionnaire for the STs.

\par \textbf{Flow of Events:}

\begin{enumerate}
    \item CO request the page that shows the applications for his internship.
    \item S\&C fetches the applications from the database.
    \item S\&C shows the applications to the CO.
    \item CO selects the STs he wants to send the questionnaire.
    \item S\&C store the selection of STs linked to the internship.
    \item CO confirm the questionnaire to be sent to the selected STs.
    \item S\&C notifies the selected STs that they have a questionnaire to fill, inherent to the CO's internship, via email.
\end{enumerate}

\par \textbf{Exit Condition:} CO receives a confirmation that S\&C has sent the questionnaire to the selected STs.

\par \textbf{Exceptions:} S\&C generated an internal error.

\par 
\begin{figure}[H]
    \centering
    \includegraphics[width=0.8\textwidth]{Images/UC_12.pdf}
    \caption{Company select applicant student for his internship use case diagram}
    \label{fig:use-case-diagram-12}
\end{figure}

\nextUseCaseID: Company Create A Complaints

\par \textbf{Actors:} Company

\par \textbf{Entry Condition:} The CO is correctly logged in. The CO had decided to create a complaint about the internship he's holding.

\par \textbf{Flow of Events:}

\begin{enumerate}
    \item CO clicks on the "Create Complaint" button.
    \item S\&C send to the CO a form to fill with the complaint details.
    \item CO fills the form and submits it.
    \item S\&C stores the complaint in the database.
    \item S\&C request to the Emailing system to send an email to the UN about a new complaint about an internship of one of it's ST.
    \item The Emailing system sends the email to the UN.
    \item S\&C notifies the CO that the complaint was successfully submitted.

    Executed When UN respond:
    \item UN acknowledges the complaint.
    \item S\&C change the status of the complaint to "Acknowledged".
\end{enumerate}

\par \textbf{Exit Condition:} CO receive the confirmation of the complaint being saved correctly

\par \textbf{Exceptions:}  S\&C generated an internal error.

\begin{figure}[H]
    \centering
    \includegraphics[width=1.0\textwidth]{Images/UC_13.pdf}
    \caption{Company creates a complaint use case diagram}
    \label{fig:use-case-diagram-13}
\end{figure}

\nextUseCaseID: Company Answers Questionnaire At Internship completion

\par \textbf{Actors:} Company

\par \textbf{Entry Condition:} The CO is correctly logged in. The internship has ended. The CO has received a notification to fill the questionnaire.

\par \textbf{Flow of Events:}

\begin{enumerate}
    \item CO clicks on the "Fill Questionnaire" button.
    \item S\&C fetch the questionnaire from the database.
    \item S\&C shows the questionnaire to the CO.
    \item CO fills each question and submits it.
    \item S\&C stores the questionnaire in the database.
    \item S\&C notifies the UN that the questionnaire was successfully submitted.
    \item S\&C feeds the answers of the questionnaire to the suggestion system to improve it's answare on how to improve a CV or an internship advertisement.
\end{enumerate}

\par \textbf{Exit Condition:} CO receive the confirmation of the questionnaire being saved correctly

\par \textbf{Exceptions:}  S\&C generated an internal error.

\begin{figure}[H]
    \centering
    \includegraphics[width=1.0\textwidth]{Images/UC_14.pdf}
    \caption{Company answers questionnaire at internship completion use case diagram}
    \label{fig:use-case-diagram-14}
\end{figure}

\nextUseCaseID: University Reviews and Handles Complaint

\par \textbf{Actors:} University

\par \textbf{Entry Condition:} The UN is correctly logged in. The UN has a list of complaints to review.

\par \textbf{Flow of Events:}

\begin{enumerate}
    \item UN clicks on the "Complaints" button.
    \item S\&C fetch the complaints from the database.
    \item S\&C shows a preview of the complaints to the UN.
    \item UN selects a complaint to review.
    \item S\&C fetch the complaint details from the database.
    \item S\&C shows the complaint details to the UN.
    \item UN review the complaint and writes a response and decide if the internship need to be suspended.
    \item S\&C update the status of the complaint with the UN response.
    \item S\&C suspend the internship if the UN decided to do so.
    \item S\&C notifies the UN that the complaint was successfully reviewed.
\end{enumerate}

\par \textbf{Exit Condition:} UN receive the confirmation of the complaint being reviewed correctly

\par \textbf{Exceptions:}  S\&C generated an internal error.


\begin{figure}[H]
    \centering
    \includegraphics[width=1.0\textwidth]{Images/UC_16.pdf}
    \caption{University reviews and handles complaint use case diagram}
    \label{fig:use-case-diagram-16}
\end{figure}
