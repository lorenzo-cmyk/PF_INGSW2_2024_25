\chapter{Implementation, Integration and Testing}
\label{ch:implementation-integration-and-testing}%

\section{Overview and Implementation Plan}
\label{sec:overview-and-implementation-plan}%

\par In this chapter, we will discuss the implementation of the S\&C Web Application, the integration and the test
strategy that will be used to ensure the quality of the software. In general, the methodology used to implement the
S\&C Web Application is the Bottom-Up approach, which is a software development approach that starts by implementing
the lower-level modules first, and then integrating them to create higher-level modules.

\par By using this approach, we can ensure that the lower-level modules are working correctly before integrating them into
higher-level modules. Each module will require a Driver to test the module. The Driver will simulate the behavior of
the higher-level module and will call the lower-level module to test its functionality by making calls of the interfaces
given by the lower-level module.
The Bottom-Up strategy grants a step by step integration of the modules and ensures that the software is working correctly
thanks to the testing of each module, that will make easier to identify and fix the bugs of the different modules before
integrating them into more complex ones. This strategy also allows the development team to work on different modules
simultaneously, which can speed up the development process.


\section{Features Identifications}
\label{sec:features-identifications}%

\par The S\&C Web Application is a complex software that has different features that are going to be used by the different
users. The features are divided into three main categories: the CO Features, the ST Features, and the UN Features. Each
category has different features that are going to be used by the different users. The features are identified as follows:

\par \textbf{CO Features}: This set of features is used by the COs to manage their profile, the internships
advertisement that they will create, their editing, the creation of the questionnaires, the management of the applicants
lists and the management of the internships.

\begin{itemize}
    \item \textbf{[CO.1] Profile Management}: This feature allows the COs to manage their profile, by changing their
    CO's information, such as the description provided and the CO's logo.
    \item \textbf{[CO.2] COs' Questionnaires Management}: This feature allows the COs to create, edit and delete the 
    questionnaires that they will send to the applicants that have applied to their internships.
    \item \textbf{[CO.3] Internship Advertisement Management}: This feature allows the COs to create, edit and delete
    the internships advertisements.
    \item \textbf{[CO.4] Internship Progress Management}: This feature allows the COs to see the applicants that have
    applied to their internships, to visualize the applicants' information, to select the applicants that they want to
    send the interview questionnaires and to set the deadlines for the applicants to answer the questionnaires.
    \item \textbf{[CO.5] System's Questionnaires Answering}: This feature allows the COs to answer the feedback 
    questionnaires that the system will send to them after the internships have ended.
\end{itemize}

\par \textbf{ST Features}: This set of features is used by the STs to manage their profile, to apply to the 
internships and to answer the questionnaires sent by the COs.

\begin{itemize}
    \item \textbf{[ST.1] Profile Management}: This feature allows the STs to manage their profile, by changing their
    ST's information, and to load their CV.
    \item \textbf{[ST.2] Internship Advertisement Application}: This feature allows the STs to apply to the internships
    advertisements that they are interested in.
    \item \textbf{[ST.3] COs' Questionnaires Answering}: This feature allows the STs to answer the questionnaires sent by
    the COs and to see the deadlines for the questionnaires.
    \item \textbf{[ST.4] Internship Progression}: This feature allows the STs to see the status of the internships
    advertisement that they have applied to.
    \item \textbf{[ST.5] System's Questionnaires Answering}: This feature allows the STs to answer the feedback 
    questionnaires that the system will send to them after the internships have ended.
\end{itemize}

\par \textbf{UN Features}: This set of features is used by the UN's staff to visualize the information of the
internships to which it's students are involved, the contracts that the UN has with S\&C, and the ability to block the
COs that are not following the terms of the application.

\begin{itemize}
    \item \textbf{[UN.1] Internships Information Visualization}: This feature allows the UN's staff to see the
    information and the status of the internships to which it's students are involved.
    \item \textbf{[UN.2] Contracts Visualization}: This feature allows the UN's staff to see the contracts that the UN
    has with S\&C.
    \item \textbf{[UN.3] COs' Blocking}: This feature allows the UN's staff to block the COs.
\end{itemize}


\par This feature identification is useful to identify what the different users can do in the Web Application. 
Now it's going to be discussed how the different features are going to be implemented based on the architecture given in
chapter \ref{ch:architectural-design}:

\par \textbf{[F1] Login Features}: 

\begin{itemize}
    \item \textbf{[F1.1] Login by SSOs}: This feature allows the STs and the UN's staff to login into the Web Application
    using their SSOs.
    \item \textbf{[F1.2] Login by Email and Password}: This feature allows the COs to login into the Web Application
    using the username and password provided by S\&C staff
\end{itemize}

\par \textbf{[F2] Internship Management Features}:

\begin{itemize}
    \item \textbf{[CO.3] Internship Advertisement Management}: This feature allows the COs to create, edit and delete
    the internships advertisements.
    \item \textbf{[CO.4] Internship Progress Management}: This feature allows the COs to see the applicants that have
    applied to their internships, to visualize the applicants' information, to select the applicants that they want to
    send the interview questionnaires and to set the deadlines for the applicants to answer the questionnaires.
    \item \textbf{[ST.2] Internship Advertisement Application}: This feature allows the STs to apply to the internships
    advertisements that they are interested in.
    \item \textbf{[ST.4] Internship Progression}: This feature allows the STs to see the status of the internships
    advertisement that they have applied to.
    \item \textbf{[UN.1] Internships Information Visualization}: This feature allows the UN's staff to see the
    information and the status of the internships to which it's students are involved.
\end{itemize}


\par \textbf{[F3] Profile Management Features}:

\begin{itemize}
    \item \textbf{[CO.1] Profile Management}: This feature allows the COs to manage their profile, by changing their
    CO's information, such as the description provided and the CO's logo.
    \item \textbf{[ST.1] Profile Management}: This feature allows the STs to manage their profile, by changing their
    ST's information, and to load their CV.
\end{itemize}

\par \textbf{[F4] Questionnaires Management Features}:

\begin{itemize}
    \item \textbf{[CO.2] COs' Questionnaires Management}: This feature allows the COs to create, edit and delete the 
    questionnaires that they will send to the applicants that have applied to their internships.
    \item \textbf{[CO.5] System's Questionnaires Answering}: This feature allows the COs to answer the feedback 
    questionnaires that the system will send to them after the internships have ended.
    \item \textbf{[ST.3] COs' Questionnaires Answering}: This feature allows the STs to answer the questionnaires sent by
    the COs and to see the deadlines for the questionnaires.
    \item \textbf{[ST.5] System's Questionnaires Answering}: This feature allows the STs to answer the feedback 
    questionnaires that the system will send to them after the internships have ended.

\end{itemize} 

\par \textbf{[F5] Complaint Features}: This set of features is used by the COs and the STs to make complaints about the
internships, and allow the UN's staff to see the complaints and to take action about it.

\begin{itemize}
    \item \textbf{[F5.1] Complaint Creation}: This feature allows the COs and the STs to create complaints about the
    internships.
    \item \textbf{[F5.2] Complaint Visualization}: This feature allows all the users involved to see the complaint 
    created.
    \item \textbf{[F5.3] Complaint Resolution}: This feature allows the UN's staff to take action about the complaints
    and eventually to suspend the internship.
\end{itemize}

\par \textbf{[F6] Notification Features}: This set of features is to send notifications to the user, like 
that they received a questionnaires to which they need to answer, or that a new complaint has been created about an internship
that they are involved in.

\par \textbf{[F7] Recommendation Features}: This set of features is to recommend internships to the STs based on their
CVs and to the COs to recommend STs based on the internships that they have created.

\par \textbf{[F8] Suggestions Features}: This set of features is to suggest to the COs how to improve their internships
applications to make them more attractive to the STs, to the STs how to make their CVs more attractive to the COs

\par \textbf{[F9] Visualization Features}: This set of features is to visualize all the information that a user needs,
the notifications that they have received, the search engine with it's filter.


\section{Integration Strategy}
\label{sec:integration-strategy}%

\par The integration strategy and the testing will start as soon as the DBMS and the host server are built and ready 
to be used. The integration will be done by using the Bottom-Up approach an so the first module that will be implemented
and tested will be the Query Manager module, that is the module that will return the fundamental classes that are going to
be used by the other modules. So the Query Manager module is containing and will generate object of the classes that are
part of the Model in the MVC architecture. The Query Manager module will be tested by using a proper Driver that will 
simulate the behavior of the other modules that will use the Query Manager module.  



